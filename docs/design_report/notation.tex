% Here, you can define your own macros. Some examples are given below.

\definecolor{red}{RGB}{208,25,25}
\definecolor{orange}{RGB}{242,113,28}
\definecolor{yellow}{RGB}{251,189,8}
\definecolor{olive}{RGB}{181,204,24}
\definecolor{green}{RGB}{33,186,69}
\definecolor{teal}{RGB}{0,181,173}
\definecolor{blue}{RGB}{33,133,208}
\definecolor{violet}{RGB}{100,53,201}
\definecolor{purple}{RGB}{163,51,200}
\definecolor{pink}{RGB}{224,57,151}
\definecolor{brown}{RGB}{165,103,63}
\definecolor{gray}{RGB}{118,118,118}
\definecolor{black}{RGB}{27,28,29}

% \newcommand{\defeq}{\coloneq}
\newcommand{\defeq}{:=}
\newcommand{\ok}{\mathsf{typed}}
\newcommand{\dom}{\operatorname{dom}}
\newcommand{\defn}{\operatorname{def}}
\newcommand{\free}{\operatorname{free}}
\newcommand{\holes}{\operatorname{holes}}
\newcommand{\subst}[3]{\{{#2}/{#1}\}\,{#3}}
\newcommand{\inter}[1]{\llbracket{#1}\rrbracket}
\newcommand{\trans}[1]{\{{#1}\}}
\newcommand{\llet}[3]{[{#1}\defeq{#2}]\,{#3}}
\newcommand{\ppi}[3]{\Pi{#1} : {#2}.\,{#3}}
\newcommand{\lam}[2]{\lambda{#1}.\,{#2}}
\newcommand{\app}[2]{{#1}\,{#2}}
\newcommand{\sig}[3]{\Sigma{#1} : {#2}.\,{#3}}
\newcommand{\pair}[2]{\langle{#1}, {#2}\rangle}
\newcommand{\fst}{\operatorname{\mathsf{fst}}}
\newcommand{\snd}{\operatorname{\mathsf{snd}}}
\newcommand{\unit}{\top}
\newcommand{\sstar}{\langle\rangle}
\newcommand{\shade}[1]{\begingroup\setlength\fboxsep{1pt}\colorbox{gray!25}{$\vphantom{()}#1$}\endgroup}

\newcommand{\termo}{\ \mathsf{term}}
\newcommand{\ctxo}{\ \mathsf{ctx}}
\newcommand{\vdasho}{\vdash}
\newcommand{\termd}{\ \mathsf{term}_{\defeq}}
\newcommand{\ctxd}{\ \mathsf{ctx}_{\defeq}}
\newcommand{\envd}{\ \mathsf{env}_{\defeq}}
\newcommand{\vdashd}{\vdash_{\defeq}}
\newcommand{\termh}{\ \mathsf{term}_{?}}
\newcommand{\ctxh}{\ \mathsf{ctx}_{?}}
\newcommand{\envh}{\ \mathsf{env}_{?}}
\newcommand{\vdashf}{\longrightarrow}
\newcommand{\vdashs}{\Longrightarrow}
\def\defaultProofSkipAmount{\vskip .6em}
\def\proofSkipAmount{\defaultProofSkipAmount}

% Fix parskip + amsthm
% See: https://tex.stackexchange.com/questions/192722/a-problem-with-amsthm-and-positive-parskip
\let\oldproof\proof
\def\proof{\oldproof\unskip}

\theoremstyle{plain}
\newtheorem{proposition}{Proposition}[chapter]

\theoremstyle{definition}
\newtheorem{definition}[proposition]{Definition}

\theoremstyle{definition}
\newtheorem*{example}{Example}
